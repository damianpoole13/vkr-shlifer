\section{Анализ предметной области}
\subsection{Характеристика графического пользовательского интерфейса}

Первая концепция графического пользовательского интерфейса была создана ученым Дагом Энгельбартом в 1960-х годах, затем данную концепцию переняли ученые из лаборатории Xerox Parc и в последствии появилась система  WIMP (Windows, Icons, Menus, Pointers) в 1973 г., но коммерческое воплощение данная концепция получила только в 1984 благодаря продукции Apple Computers. Сейчас мы можем видеть GUI во всех операционных системах.

Термин "<графическеский интерфейс пользователя"> (ГИП) означает использование графических элементов для взаимодействия с компьютером \cite{gui}. Функциональные элементы интерфейса должны отображаться в соответствии с их назначением и свойствами. Сегодня практически невозможно найти программу без использования меню и кнопок, благодаря которым пользователь может взаимодействовать с ПК. 

При проектировании ГИП используется концепция «Do what I mean» (DWIM). Согласно концепции требуется, чтобы система пыталась предугадать намерения пользователя, автоматически исправляя тривиальные ошибки, а не слепо выполняя явные, но потенциально неверные действия пользователя.

ГИП предполагает построение окна методом добавления элементов для взаимодействия (Control) и обработчиков событий (Event).     

Графические интерфейсы можно классифицировать следующим образом:

\begin{enumerate}
\item Простые: стандартные экранные формы и элементы интерфейса, предоставляемые самой GUI.
\item Истинно-графические, двумерные: нестандартные элементы интерфейса и оригинальные метафоры, созданные с использованием собственных возможностей приложения или сторонней библиотеки.
\item Трехмерные.
\end{enumerate}

В современном мире все большее количество приложений оснащаются графическими интерфейсами. Применение графических интерфейсов программ (ГИП) способствует улучшению наглядности, удобства использования и интуитивного понимания функционала приложений.  

\subsection{Характеристика фреймворков}

История появления фреймворков для создания графического пользовательского интерфейса (GUI) началась в 1980-х годах с развитием персональных компьютеров и появлением операционных систем, поддерживающих GUI, таких как Macintosh System Software и Windows.

Программисты столкнулись с необходимостью создания удобных и интуитивно понятных пользовательских интерфейсов для своих приложений, и в ответ на это начали появляться различные библиотеки и фреймворки, упрощающие и ускоряющие процесс создания GUI-приложений.

Со временем фреймворки становились все более мощными и универсальными, позволяя разработчикам создавать кросс-платформенные приложения с современными и стильными интерфейсами. Количество и разнообразие фреймворков для GUI продолжает расти, отвечая на потребности разработчиков в удобных и эффективных инструментах для создания приложений.

Фреймворк \cite{framework} - это набор структурированных и готовых к использованию компонентов, библиотек и инструментов, который предоставляет разработчику рамочную структуру для создания приложения. Фреймворк облегчает процесс разработки, предоставляя готовые решения для часто встречающихся задач, таких как управление данными, обработка событий, взаимодействие с пользовательским интерфейсом и многое другое.

Использование фреймворка позволяет ускорить процесс разработки, повысить стабильность и масштабируемость приложения, а также упростить поддержку и обновление кода. Фреймворк также обеспечивает стандартизацию процесса разработки, что помогает разработчикам лучше ориентироваться в коде и улучшить его качество.

Фреймворки могут быть общими, предназначенными для разработки различных видов приложений, или специализированными, ориентированными на конкретные типы приложений или технологии.

\subsection{Сравнение популярных GUI фреймворков}

\subsubsection {Qt}
Qt \cite{Qt} является мощным фреймворком для разработки кросс-платформенных десктопных приложений с графическим интерфейсом. Он предоставляет широкий набор компонентов и инструментов для создания современных и качественных интерфейсов.

Плюсы:
\begin{itemize}
\item поддержка множества платформ (Windows, macOS, Linux, Android, iOS);
\item богатый набор виджетов и инструментов;
\item высокая производительность;
\item активное сообщество и хорошая документация.
\end{itemize}

Минусы:
\begin{itemize}
\item лицензирование может быть дорогим для коммерческих приложений;
\item может иметь крутой порог вхождения для начинающих.
\end{itemize}

\subsubsection {Electron}
Electron \cite{electron} позволяет разработчикам создавать кросс-платформенные десктопные приложения с использованием веб-технологий, таких как HTML, CSS и JavaScript. Он обеспечивает возможность упаковки веб-приложений в исполняемые файлы для различных операционных систем.

Плюсы:
\begin{itemize}
\item использует веб-технологии (HTML, CSS, JavaScript), что упрощает разработку для веб-разработчиков;
\item кросс-платформенность (Windows, macOS, Linux);
\item большое сообщество и множество библиотек.
\end{itemize}

Минусы:
\begin{itemize}
\item высокое потребление ресурсов и большой размер конечного приложения;
\item меньшая производительность по сравнению с нативными решениями.
\end{itemize}

\subsubsection {JavaFX}
JavaFX \cite{javafx} - это фреймворк, предоставляемый Java для создания кросс-платформенных десктопных и веб-приложений с графическим интерфейсом. Он поддерживает различные стили и эффекты для дизайна интерфейса.

Плюсы:
\begin{itemize}
\item хорошо интегрируется с экосистемой Java;
\item поддержка мультимедийных и 3D-графических возможностей;
\item кросс-платформенность (Windows, macOS, Linux).
\end{itemize}
Минусы:
\begin{itemize}
\item менее популярный по сравнению с другими решениями, что может усложнить поиск ресурсов и поддержки;
\item весомый размер приложений по сравнению с нативными решениями.
\end{itemize}

\subsubsection {GTK+}
GTK \cite{gtk} - это набор библиотек и инструментов для создания графического пользовательского интерфейса в Linux и других UNIX-подобных операционных системах. Он предоставляет различные виджеты и стили для разработки качественных интерфейсов.

Плюсы:
\begin{itemize}
\item открытый исходный код и бесплатность;
\item хорошая поддержка Linux;
\item легкость интеграции с языками, такими как Python.
\end{itemize}

Минусы:
\begin{itemize}
\item меньшая поддержка Windows и macOS по сравнению с другими фреймворками;
\item интерфейс может выглядеть не так современно.
\end{itemize}

\subsubsection {WPF (Windows Presentation Foundation)}
WPF \cite{wpf} предоставляет инструменты для создания десктопных приложений на платформе Windows с использованием .NET Framework. Он поддерживает создание богатых и интерактивных пользовательских интерфейсов с помощью XAML (Extensible Application Markup Language).

Плюсы:
\begin{itemize}
\item отличная интеграция в экосистему Windows;
\item богатый набор функций для создания сложных интерфейсов;
\item поддержка XAML для декларативного описания интерфейсов.
\end{itemize}

Минусы:
\begin{itemize}
\item привязанность к платформе Windows;
\item крутая кривая обучения для новичков.
\end{itemize}

\subsubsection {Перспектива развития}
Проанализировав самые популярные фреймворки для создания графических интерфейсов, можно сделать вывод, что возникают сложности с лицензированием, высокого потребления ресурсов и меньшей производительности, крутого порога вхождения и усложненного поиска ресурсов.

К перспективам развития GUI фрейворков можно отнести:
\begin{enumerate}
\item Улучшение производительности: разработчики фреймворков будут стремиться увеличить скорость работы приложений, оптимизировать процессы отрисовки интерфейсов и уменьшить нагрузку на процессор.
\item Поддержка новых технологий: фреймворки будут активно интегрировать новые технологии, такие как искусственный интеллект, виртуальная реальность и др., для создания интерактивных и инновационных пользовательских интерфейсов.
\item Улучшение удобства использования: фреймворки будут работать над улучшением удобства использования пользовательского интерфейса, добавляя новые функции, улучшая навигацию и дизайн, а также упрощая процесс создания интерфейсов для разработчиков.
\item Развитие мобильной адаптации: с увеличением количества мобильных устройств, фреймворки будут активно развивать механизмы адаптации пользовательских интерфейсов под различные размеры экранов и разрешения.
\item Повышение безопасности: разработчики фреймворков будут усиливать уровень безопасности пользовательских интерфейсов, защищая их от взломов, утечек данных и других угроз.
\item Поддержка кроссплатформенности: фреймворки будут продолжать развивать возможности кроссплатформенной разработки, позволяя создавать приложения с единным кодом для различных операционных систем и устройств.
\item Интеграция с новыми технологиями интерфейсов: разработчики фреймворков будут работать над интеграцией новых технологий интерфейсов, таких как голосовые команды, жесты и распознавание лиц, для создания более удобного и интуитивного пользовательского опыта.
\end{enumerate}
