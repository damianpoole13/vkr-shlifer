\addcontentsline{toc}{section}{СПИСОК ИСПОЛЬЗОВАННЫХ ИСТОЧНИКОВ}

\begin{thebibliography}{20}
	\bibitem{interface} Купер Алан. Интерфейс. Основы проектирования взаимодействия / А. Купер. – Санкт-Петербург: Питер, 2018. – 720 с. – ISBN 978-5-4461-0877-0. – Текст~: непосредственный.
	\bibitem{gui} Машин В. А. Проектирование и дизайн пользовательского интерфейса / В.А. Машин, А.К. Гультяев. – Санкт-Петербург: Корона Принт, 2010. – 352 с. – ISBN 978-5-7931-0814-0. – Текст~: непосредственный.
	\bibitem{framework} Gregory F. Rogers. Framework-Based Software Development in C++  / Gregory F. Rogers. – London: Pearson, 2008. – 400 с. – ISBN 978-0135333655. – Текст~: непосредственный.
	\bibitem{Qt} Прохоренок Николай. Qt 6. Разработка оконных приложений на C++ / Прохоренок Н.А. – Санкт-Петербург: БХВ, 2022. – 512 с. – ISBN 978-5-9775-1180-3. – Текст~: непосредственный.
	\bibitem{electron} Скотт Адам Д. Разработка на JavaScript. Построение кроссплатформенных приложений с помощью GraphQL, React, React Native и Electron / Скотт А. Д. – Санкт-Петербург: Питер, 2021. – 320 с. – ISBN 978-5-4461-1462-7. – Текст~: непосредственный.
	\bibitem{javafx} Прохоренок Николай. JavaFX / Прохоренок Н.А. – Санкт-Петербург: БХВ, 2019. – 768 с. – ISBN 978-5-9775-4072-8. – Текст~: непосредственный.
	\bibitem{gtk} Андрей Костельцов. GTK+. Разработка переносимых графических интерфейсов / Костельцов А.В. – Санкт-Петербург: БХВ, 2002. – 362 с. – ISBN 5-94157-161-5. – Текст~: непосредственный.
	\bibitem{wpf} Мак-Дональд Мэтью. WPF: Windows Presentation Foundation в .NET 4.5 с примерами на C{\#} 5.0 для профессионалов. / Мак-Дональд М. – Москва: Вильямс, 2013. – 1024 с. – ISBN 978-5-8459-1854-3. – Текст~: непосредственный.
	\bibitem{uml} Максимчук Р.А. UML для простых смертных /Максимчук  Р.А., Нейбург Э.Дж. – Москва: Лори, 2024. – 300 с. – ISBN 978-5-85582-434-6. – Текст~: непосредственный.
	\bibitem{gl} Гинсбург Дэн. OpenGL ES 3.0. Руководство разработчика / Гинсбург Д. – Санкт-Петербург: ДМК Пресс, 2015. – 449 с. – ISBN 978-5-97060-256-0. – Текст~: непосредственный.
	\bibitem{vulkan} Селлерс Грэхем. Vulkan. Руководство разработчика / Селлерс Г. – Санкт-Петербург: ДМК Пресс, 2017. – 394 с. – ISBN 978-5-97060-486-1. – Текст~: непосредственный.
    \bibitem{OOP} E. Balagurusamy. Object Oriented Programming With C++ / Balagurusamy E.  – New York Sity : McGraw-Hill Education (India) Pvt Limited, 2008 – 637 с. – ISBN 978-0070669079. – Текст~: непосредственный.
    \bibitem{c++} Paul Deitel C++ How to Program / Deitel P., Deitel H. – London: Pearson, 2016. – 1080 с. – ISBN 978-0134448237. – Текст~: непосредственный.
    \bibitem{recommendations} Мейерс Скотт. Эффективный и современный С++:42 рекомендации по использованию С++11 и С++14 / Мейерс С. – Москва~: Вильямс, 2019. – 304 с. – ISBN 978-5-907114-67-8. – Текст~: непосредственный.
	\bibitem{problems} Дьюхерст Стивен. Скользкие места С++. Как избежать проблем при проектировании и компиляции ваших программ / Дьюхерст С. – Москва~: ДМК Пресс, 2017. – 264 с. – ISBN 978-5-97060-475-5. – Текст~: непосредственный.
	\bibitem{beauty} Грегори Кейт. Красивый C++: 30 главных правил чистого, безопасного и быстрого кода / Кейт Г., Дэвидсон Дж. Г. – Санкт-Петербург~: Питер, 2023. – 368 с. – ISBN 978-5-4461-2272-1. – Текст~: непосредственный.
	\bibitem{creative} Спрол Антон. Думай как программист. Креативный подход к созданию кода. C++ версия / Спрол А. – Москва~: Эксмо, 2018. – 272 с. – ISBN 978-5-04-089838-1. – Текст~: непосредственный.
	\bibitem{boost} Полухин Антон. Разработка приложений на С++ с использованием Boost. Рецепты, упрощающие разработку вашего приложения / Полухин А. - Санкт-Петербург : ДМК Пресс, 2020. - 346 c. -ISBN 978-5-97060-868-5. – Текст~: непосредственный.
	\bibitem{chainik} Шилдт Герберт. C++ для начинающих / Шилдт Г. – Санкт-Петербург: Питер, 2024. – 608 с. – ISBN 978-5-4461-1821-2. – Текст~: непосредственный.
	\bibitem{practika} Евдокимов Петр. C++ на примерах. Практика, практика и только практика / Евдокимов П. В., Орленко П. А. – Санкт-Петербург: Наука и техника, 2022. – 288 с. – ISBN 978-5-94387-772-8. – Текст~: непосредственный.
	\bibitem{history} Страуструп Бьёрн. Дизайн и эволюция языка-С++. Второе издание / Страуструп Б. – Санкт-Петербург: ДМК Пресс, 2016. – 448 с. – ISBN 978-5-97060-419-9. – Текст~: непосредственный.
	\bibitem{visual} Пахомов Б.И. C/C++ и MS Visual C++ 2010 для начинающих / Пахомов Б.И. – Санкт-Петербург: БХВ, 2011. – 736 с. – ISBN 978-5-9775-0599-4. – Текст~: непосредственный.
	\bibitem{student} Побегайло А. П.  C/С++ для студента / Побегайло А. П. – Санкт-Петербург: БХВ, 2006. – 525 с. – ISBN 978-5-94157-647-1. – Текст~: непосредственный.
\end{thebibliography}
