\section*{ЗАКЛЮЧЕНИЕ}
\addcontentsline{toc}{section}{ЗАКЛЮЧЕНИЕ}

В ходе исследования было выявлено, что проектирование фреймворка для создания графического пользовательского интерфейса играет ключевую роль в обеспечении удобства использования программного продукта. Разработка такого фреймворка позволяет значительно упростить процесс создания интерфейсов, повысить их качество и снизить затраты на разработку. Отличительной особенностью успешного фреймворка является его гибкость, расширяемость и простота в использовании. Дальнейшее развитие в этом направлении сможет улучшить опыт пользователей и повысить эффективность разработки программных продуктов.

К особенностям данного фреймворка можно отнести встроенную поддержку локалей и цветовых тем на основе json-схем позволяет легко создавать впечатляющие многоязычные приложения с разнообразными цветовыми и визуальными темами. Еще одним приимуществом является небольшой средний размер двоичного кода, что позволило значительно уменьшить итоговый вес приложений. 

Основные результаты работы:

\begin{enumerate}
\item Проведен анализ предметной области. Выявлена необходимость использовать С++.
\item Разработана концептуальная модель фреймворка. Разработана модель данных системы. Определены требования к системе.
\item Cпроектирована программная система для создания графических пользовательских интерфейсов
\item Реализован и протестирован фреймворк. Проведено модульное и системное тестирование.
\end{enumerate}

Все требования, объявленные в техническом задании, были полностью реализованы, все задачи, поставленные в начале разработки проекта, были также решены.

Готовый рабочий проект представлен библиотекой. Фреймворк находится в публичном доступе, поскольку опубликован в сети Интернет.

Перспективой дальнейшей разработки является расширение списка поддерживаемых элементов управления и шаблонов программ.
