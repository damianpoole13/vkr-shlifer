\section*{ВВЕДЕНИЕ}
\addcontentsline{toc}{section}{ВВЕДЕНИЕ}

С развитием современных технологий и повышением требований к функциональности и производительности приложений, использование фреймворков становится все более актуальным для разработки программного обеспечения. Фреймворк – это набор готовых компонентов, библиотек и шаблонов, который позволяет ускорить процесс разработки, повысить надежность и безопасность приложения, а также облегчить его поддержку и масштабируемость.

Использование фреймворков позволяет разработчикам сосредоточиться на решении бизнес-задач, минуя необходимость повторного написания базового функционала. Это позволяет сократить время разработки, снизить затраты на обучение и поддержку разработчиков, а также повысить качество и уровень безопасности разрабатываемого приложения.

В связи с этим, фреймворки становятся неотъемлемой частью современного мира. Они обеспечивают разработчиков мощными инструментами и инфраструктурой, ускоряют процесс разработки и обеспечивают высокую гибкость и масштабируемость приложений. Именно поэтому использование фреймворков становится все более популярным среди разработчиков и компаний, стремящихся к созданию современных и эффективных приложений.

\emph{Цель настоящей работы} – разработка фреймворка для создания графического пользовательского интерфейса. Для достижения поставленной цели необходимо решить \emph{следующие задачи:}
\begin{itemize}
\item провести анализ предметной области;
\item разработать концептуальную модель фреймворка;
\item спроектировать фреймворк;
\item протестировать полученные с помощью фреймворка приложения.
\end{itemize}

\emph{Структура и объем работы.} Отчет состоит из введения, 4 разделов основной части, заключения, списка использованных источников, 2 приложений. Текст выпускной квалификационной работы равен \formbytotal{lastpage}{страниц}{е}{ам}{ам}.

\emph{Во введении} сформулирована цель работы, поставлены задачи разработки, описана структура работы, приведено краткое содержание каждого из разделов.

\emph{В первом разделе} на стадии описания технической характеристики предметной области приводится сбор информации.

\emph{Во втором разделе} на стадии технического задания приводятся требования к разрабатываемому фреймворку.

\emph{В третьем разделе} на стадии технического проектирования представлены проектные решения для фреймворка.

\emph{В четвертом разделе} приводится список классов и их методов, использованных при разработке фреймворка, производится тестирование разработанного фреймворка.

В заключении излагаются основные результаты работы, полученные в ходе разработки.

В приложении А представлен графический материал.
В приложении Б представлены фрагменты исходного кода. 
