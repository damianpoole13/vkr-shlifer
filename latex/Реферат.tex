\abstract{РЕФЕРАТ}

Объем работы равен \formbytotal{lastpage}{страниц}{е}{ам}{ам}. Работа содержит \formbytotal{figurecnt}{иллюстраци}{ю}{и}{й}, \formbytotal{tablecnt}{таблиц}{у}{ы}{}, \arabic{bibcount} библиографических источников и \formbytotal{числоПлакатов}{лист}{}{а}{ов} графического материала. Количество приложений – 2. Графический материал представлен в приложении А. Фрагменты исходного кода представлены в приложении Б.

Перечень ключевых слов: фреймворк, графический интерфейс, интерфейс, библиотека, классы, элементы управления, событие, окно, компонент, метод.

Объектом разработки является фреймворк для создания элементов графического пользовательского интерфейса.

Целью выпускной квалификационной работы является создание удобной библиотеки для создания элементов пользовательского интерфейса.

В процессе создания фреймворка были выделены основные классы путем создания информационных блоков, использованы функции и методы модулей, обеспечивающие работу с сущностями предметной области.

При разработке фреймворка использовалась среда разработки "<Microsoft Visual Studio">.

\selectlanguage{english}
\abstract{ABSTRACT}
  
The volume of work is \formbytotal{lastpage}{page}{}{s}{s}. The work contains \formbytotal{figurecnt}{illustration}{}{s}{s}, \formbytotal{tablecnt}{table}{}{s}{s}, \arabic{bibcount} bibliographic sources and \formbytotal{числоПлакатов}{sheet}{}{s}{s} of graphic material. The number of applications is 2. The graphic material is presented in annex A. The layout of the site, including the connection of components, is presented in annex B.

List of keywords: framework, GUI, library, classes, control, event, window, interface, method.

The object of development is a framework for creating elements of graphical user interface.

The objective of the final qualification work is to create a convenient library for creating user interface elements.

In the process of creating the framework, the main classes were identified by creating information blocks, functions and methods of modules that provide work with the entities of the subject area were used.

The development environment "<Microsoft Visual Studio"> was used during the framework development.
\selectlanguage{russian}
